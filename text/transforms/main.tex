\section{Language Transforms\timeestimation{25h}} % (fold)
\label{sec:transforms}
In this chapter, we'll discuss how we can use the notion of a data 
representation of a program to get a standard toolchain.
\subsection{Language Subsets} % (fold)
\label{sub:Language Subsets}
\paragraph{{\tt C++}  and {\tt C} } % (fold)
\label{par:Cpp and C}
\begin{defn}
	Let {\tt A} and {\tt B} be two languages such that each valid {\tt B} 
	program is also a valid {\tt A} program. Further 
	$\forall b\in B\forall d\in Dat(B): \interpret{b}_A(d) =
	\interpret{b}_B(d)$
\end{defn}

% paragraph Cpp and C (end)
% subsection Language Subsets (end)
\subsection{Interpreter} % (fold)
\label{sub:Interpreter}

% subsection Interpreter (end)
\subsection{Compiler} % (fold)
\label{sub:Compiler}
\paragraph{How the {\tt gcc} is ported} % (fold)
\label{par:gcc}
\lineofthought{ Bootstrapping von {\tt gcc} }
% paragraph Wie der gcc portiert wird (end)

% subsection Compiler (end)
\subsection{Futamura Projections} % (fold)
\label{sub:Futamura}
\subsubsection{Specializer} % (fold)
\label{ssub:Specializer}
\subsubsection{Futamura Projections} % (fold)
\label{ssub:Futamura Projections}
The notion of a specializer as a transformer of source code has lead to some 
interesting observations: \lineofthought{ 
	\begin{itemize}
		\item An interpreter spec'ed with source is executable ($\rightarrow$ py2exe)
		\item A compiler is a specialized specializer with the step above.
		\item Repeat to get a compiler generator.
	\end{itemize} 
	Use types to visualize (
	$\applied{spec}: Input_1 \rightarrow \coded{\left( Input_2 \rightarrow Output \right)}$)
}

% subsubsection Futamura Projections (end)
\paragraph{The PyPy project} % (fold)
\label{par:The PyPy project}
\begin{example}
	The {\tt PyPy} project is an attempt to implement the popular 
	Python\footnote{\url{http://python.org/}} itself in a subset of Python 
	(called RPython). Since Python is an interpreted language, it would seem 
	that this approach would lead to very slow execution, but that is not the 
	case: PyPy uses Just-In-Time (JIT) specialization and compilation 
	techniques in part described in \cite{psycho}.

	While the approach described in \ref{ssub:Specializer} is understood to be 
	executed before the actual program is run, it is also possible to run it 
	in parallel to the actual computation: Now the specializer can use 
	statistical information on the values. For example, while it might not be 
	obvious from the source that a certain value is constant and therefore a 
	static specializer might fail to set in, but a dynamic specializer can 
	determine this and produce a specialized function to call.

	For a highly dynamic language like Python, it can lead to a hundredfold 
	speedup for very repetitive arithmetics\footnote{\cite{psycho}}.
\end{example}

% paragraph The PyPy project (end)

% subsubsection Specializer (end)
% subsection Futamura (end)
