\section{The {\tt WHILE} Language\timeestimation{15h}}
\label{sec:WHILE}
In the last chapter we saw an example of a function that is not computable 
with a \FOR program\footnote{The semantic function of \FOR}. However with a 
simple addition to the language, we gain all we need for a language.

The new statement is called $\mathtt{while}\: e\: \{ block \}$, and it does what one would expect:

\begin{itemize}
	\item It takes an expression $e$ and a block $block$.
	\item If the evaluation of the expression yields {\tt nil}, it does nothing.
	\item Otherwise, it executes the block and repeats this procedure.
\end{itemize}

\begin{equation}
	\interpret{\mathtt{while}\: e\: block}(c) = \begin{cases}
		c, &\text{ if }\interpret{e}=Nil\\
		(\interpret{\mathtt{while}\: e\: block} \circ \interpret{block})(c), &\text{otherwise}
	\end{cases}
\end{equation}

This new statement does not necessarily terminate, in fact {\tt WHILE 
(nil.nil) \{\}} would never halt. This means, that the semantic function 
does not give a total function back -- for some inputs the interpretation of 
the source does not halt.

Looking back at the proof of the uncomputable function in \FOR called
$inverse$, that took a program and returned the boolean inverse of that
programs output when run with itself as input. We asked what
$\interpret{inverse}(inverse)$ would be. Since it can't be $TRUE$ nor
$FALSE$, it must be $\bot$. This can also be seen in that
$\interpret{eval}(inverse.inverse)$ is just an infinite recursion. 

Of course it could be that while this proof does not work, something else might
prevent us from implementing the {\tt eval} procedure. As we will see in
~\ref{sec:self}, that is not the case.

Part of this work is an implementation of the \WHILE programming language, which can be found at \url{https://github.com/zombiecalypse/Bachelor-Thesis/wiki}.
