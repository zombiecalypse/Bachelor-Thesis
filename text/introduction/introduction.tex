\section{Introduction}
Computability and complexity are in some way the basics of what computer
science is about. Where algorithmics might answer what the currently best known
solution to a problem is, computability theory says if a solution can exist and
complexity theory gives the problem (not the concrete algorithm!) a general
assessment of its difficulty, i.e. no algorithm can solve the problem faster 
than its inert complexity.

For this one needs an idea what a problem is, and when it is solvable. A 
problem can be understood as getting some kind of input data $x\in \Input[A]$ 
and needing to produce an output value $y\in \Output[A]$, in other words: 
computing a function $f:\Input[A] \rightarrow \Output[A]$\footnote{Equivalently, 
a problem can be characterized as deciding if $x\in^? B$.}.

A problem is then considered solvable, if there is in some way a solution to 
it, but what a solution looks like, divides the minds: 

\paragraph{Turing and his machines}
Turing's approach is a very mechanical one: A machine that has memory in the 
form of an arbitrarily large tape and is in a certain state. Depending on the 
tape's content at the current position and the current state, it can write to 
the tape, change its state and move one cell to the left or to the right.

\paragraph{Kleene composes functions}
Kleene held it intuitively computable that there are some simple functions 
such as the projections, the constant functions, and the successor are 
computable and that the composition of two computable functions is again 
computable. Finally every function can be coded and the codes can then be executed.

\paragraph{Markov rewrites strings}
Markov noted that executing an algorithm is basically nothing but rewriting a 
string (the memory) according to some predefined rules.

\paragraph{Programing languages}
While each of these approaches is interesting in their own right, they do not 
reflect the intuitions of a modern computer scientist, who already knows 
about programming languages and is closer to them then to building machines 
or even mathematical constructs.

Nearly all modern languages try to capture the full spectrum of solvable 
problems. For the sake of this text, a very simple language is considered, 
which consists of elements common in modern languages.

\lineofthought{
	I need to explain stuff I'll need in the chapter \ref{sec:WHILE}, e.g. 
	turing completeness (?), what a programming language is, ...
	
	\begin{itemize}
		\item Data and Language
		\item Representation of a program vs its semantics.
		\item Turing completeness
	\end{itemize}
}
\subsection{What is a programming language?} % (fold)
\label{sub:What is a programming language}
Before we can talk about computability in programming languages, we need to 
define, what a programming language is. 

\lineofthought{
	A language is a set of lists of characters (so called words). A programming 
	language also has a semantic function 
	$\interpret{.}: A\rightarrow (\ensuremath{\mathbb{N}} \rightarrow \ensuremath{\mathbb{N}})$.
}

% subsection What is a programming language (end)
