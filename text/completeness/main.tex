\section{Turing-Completeness of a Language}
\label{sec:Turing Completeness}
\lineofthought{
	List some surprisingly turing complete things, e.g. cellular automata (Rule 110), 
	string rewriting, ...

	Churches thesis: There is no intuitive extension to {\tt  WHILE} that is 
	stronger than {\tt WHILE}.

	Approximations of results, ... also work (linear slow-down). 

	Note that when there are side effects, we might be interested that a program
	does {\em not} terminate (e.g. our OS).
}
\subsection{Why are most programming languages Turing Complete?} % (fold)
\label{sub:Why are most programming languages Turing Complete?}
\lineofthought{
	While most algorithms used today are guaranteed to stop, proving this for all
	programs of a language is often hard. Things like recursion is no longer
	possible, unbound conditional loops (`while`) don't work and there are real
	problems that can not be solved this way. 

	Compare however Coq, that is {\em not} Turing Complete, but still used.
}
% subsection Why are most programming languages Turing Complete? (end)
