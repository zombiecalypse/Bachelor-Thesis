\section{Self Interpretation\timeestimation{15h}}
\label{sec:self}
Self interpretation is the ability of an formalism to support an "universal 
mechanism", that is a program that can interpret a finite description of any 
programs in itself and apply it to input.

For computability, self interpretation can be seen as some kind of gold
standard\citationneeded. This stems from the fact, that a simpler model of
computation, primitive recursive functions, are not self interpreting.
\lineofthought{How important {\em is} self-interpretation? What are the minimal
operations that are needed besides a self-interpretor to get full-blown
turing-completeness? I conjegure something extremely simple like the base
functions of primitive recursion. This would be some support for Church thesis
from below -- every sufficiently advanced language (supports
self-interpretation) is Turing-complete. 

This is actually quite obvious by the fact, that once universality is a
primitive operation, all programming constructs are then mere transformation of
data.}


