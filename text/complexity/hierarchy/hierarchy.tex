\section{The complexity hierarchy}
We can clearly see that $\LOGSPACE\subset \PSPACE\subset\EXPSPACE$ just because of 
the functions involved, but the $\TIME$ and $\SPACE$ 
hierarchy is in fact interleaved, as the following theorems will prove.
\begin{theorem}
	\[\TIME[f] \subset \SPACE[f]\]
\end{theorem}
\begin{proof}
	On our tape, the only way to increase the number of cells used is to write 
	something in an blank cell. That we can do at most once per step.
\end{proof}

\begin{theorem}
	There are $a>0$ and $q > 1$ and $\exp(x) := a\cdot q^x$ such that:
	\[\SPACE[f] \subset \TIME[\exp\circ f]\]
\end{theorem}
\begin{proof}
	\begin{enumerate}
		\item Assume that $P\in\SPACE[f(x)]$, then $\measurespace{P}(x)\leq f(\abs{x})\forall x$.
		\item Assume that $\abs{x}$ is fixed, then in how many ways can the 
			memory arranged to hold that property? Each cell can hold any of the 
			symbols of $\Gamma$, so there are $\abs{\Gamma}^{\abs{x}}$ ways to 
			arrange that. Multiply that with the number of states that are not end 
			states and you get the number of steps after either the machine is in a 
			configuration it has seen before or goes to a new configuration -- 
			necessarily an end-state. If the turing machine is in the same 
			configuration, then it will necessarily act in precisely the same way 
			as before, get to the same state again and again, and therefore loop. 
		\item But since $\measurespace{P}(x)\leq f(\abs{x})$, 
			$\measurespace{P}(x) \neq \infty$ and therefore 
			\[\measuretime{P}(x)\leq (\abs{Q} - 2)\cdot\abs{\Gamma}^{\abs{x}}\]
	\end{enumerate}
\end{proof}

\begin{corrolary}
	\[ %\LOGSPACE\subset
		\PTIME\subset\PSPACE\subset\EXPTIME\subset\EXPSPACE \]
\end{corrolary}
