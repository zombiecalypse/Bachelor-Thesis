\section{The complexity hierarchy}
We can clearly see that $\LOGSPACE\subset \PSPACE\subset\EXPSPACE$ just because of 
the functions involved, but the $\TIME$ and $\SPACE$ 
hierarchy is in fact interleaved, as the following theorems will prove.
\begin{theorem}
	\[\TIME[f] \subset \SPACE[f]\]
\end{theorem}
\begin{proof}
	On our tape, the only way to increase the number of cells used is to write 
	something in an blank cell. That we can do at most once per step.
\end{proof}

\begin{theorem}
	\[\SPACE[f(x)] \subset \TIME[2^{f(x)}]\]
\end{theorem}
\begin{proof}
	\begin{enumerate}
			\lineofthought{Ok, here we compare two different formalisms, which is possible, 
			but not without lemata, \dots proving that the compilation has a 
		certain efficiency. Maybe I should focus on \TIME and kick \SPACE in 
	favour of non-determinism}
		\item Assume that $P\in\SPACE[f(x)]$, then $\measurespace{P}(x)\leq f(\abs{x})\forall x$.
		\item Assume that $\abs{x}$ is fixed, then in how many ways can the 
			memory arranged to hold that property? We have if $n$ variables are 
			present in the source that for all states 
			$\sum_{k=1}^n\abs{Var_k} \leq f(\abs{x})$

	\end{enumerate}
	\end{proof}
