\section{Complexity in Languages\timeestimation{30h}}
\label{sec:complexity}
While most languages we encounter are turing-complete, that does not 
need to be the case and in fact some non-turing-complete languages have very 
interesting properties. In this chapter, I'll discuss some variations of the 
languages seen so far that relate to special classes of complexity and the 
notion of being complex relative to a language.
\subsection{Complexity by Dialect} % (fold)
\label{sub:Complexity by Dialect}
\subsubsection{$P$ complexity} % (fold)
\label{ssub:P-complexity}

% subsubsection P-complexity (end)
\subsubsection{$NP$ complexity} % (fold)
\label{ssub:NP-complexity}

% subsubsection NP-complexity (end)
\subsubsection{$P \neq NP$ from a Linguistic Point of View} % (fold)
\label{ssub:PNP from a Linguistic Point of View}
\lineofthought{
  Um zu zeigen, dass $NP \neq P$ ist, muss nun gezeigt werden, dass sich das 
  Kommando {\tt CHOICE} nicht in die $P$ Sprache übersetzen lässt, ohne den 
  Quelltext übermässig zu verlängern.
}

% subsubsection $P \neq NP$ from a Linguistic Point of View (end)

% subsection Complexity by Dialect (end)
