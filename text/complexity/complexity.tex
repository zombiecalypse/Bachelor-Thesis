As we saw, not all problems are solvable, but it seems not too farfetched to
say that a problem not being solvable by any algorithm is seldom a concern for
most applications. Being solvable in a reasonable amount of time and space
however can very easily become a problem. What use is a program that solves our
problem, but takes hundreds of years to complete for any reasonable input? This
is where the field of complexity theory sets in: it characterises the solvable
problems by their "difficulty".

The most intuitive measure of difficulty is arguably time-complexity, i.e. how
long any algorithm will take to solve the problem in the worst case. Also
space-complexity needs to be considered, i.e. how much storage one needs to
complete the computation.

Since bigger problems are harder to solve and bigger problems need more data to
describe it fully, we analyse the asymptotic complexity in the size of the
input, i.e. how the time and space needed develop as the input grows towards
infinity. The most important distinctions here will be if there is a linear
relation or a polynomial of higher order - or something worse, e.g. exponential
time/space.

Another important classification is that of non-determinism: if we could guess
during the computation, how would that affect the complexity of the problem?
Could we solve things faster, if we only guessed good enough? Surprisingly the
answer is that we are not sure (even though it is strongly suspected that being
a good guesser really helps).

As in computability with the Church's thesis, there is the question of what is
intuitively computable in a certain bound and even more than in computability,
this question seems to rely on the chosen formalism, but we will see that the
distinctions are quite robust. Still the question remains, if some
extraordinary feat of engineering can build a machine that is much faster than
our WHILE language.

\subsection{Time-complexity}
We will denote the time that a program $P$ runs on an input $x$ by $T_P(x)$. To
decide what that means, let's consider how long statements take to evaluate: we
have only three kind of statements, assignment, if and while. Assigning a value
can be thought of as writing an address, which can be done in one time-step. Of
course, the right hand side might take some time to evaluate, so $T_{X := E} = 1
+ T_E$. If statements take one step to check if the value is nil and the
evaluation, likewise for while. Also depending on the runtime values, the block
its evaluated zero or more times.

As for expressions, nil takes no time at all, since its not computed and all
other expressions take one step plus the evaluation of subexpressions.

Since the halting problem is not solvable, we actually need to execute the
program to know how menu steps it will take. Otherwise we could just check if
the number of steps is finite to show that P halts.

Using this, we can show that $T_{a+b}$ is $\log(a+b)$: ...
Now however that this uses that a and b are written in binary, were they for
example coded in unary, the algorithm would take a+b steps. 
