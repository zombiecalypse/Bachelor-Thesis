\section{Complexity in Languages\timeestimation{30h}}
\label{sec:complexity}
While most languages we encounter are turing-complete, that does not need
to be the case. In this chapter, we will discuss the notion of a problem 
description as a programming language.

\subsection{Problem descriptions}
There are many different kinds of problems in computer science, for example 
sorting a list or determining if a boolean formula with variables is always
true. Nevertheless, these kind of problems are never singular -- we would not 
want an algorithm\footnote{Most definitions would not even allow this as an 
algorithm}, that could reliably sort the list $4, 8, 2, -9$, but one that 
could sort {\em all} lists, no matter the size. This means of course, that 
algorithms need data as input to describe the problems they need to solve. 
This data is then called the {\em problem description}. 

Looking back at the beginning, the semantic function was introduced to give 
meaning to data. Since then, we primarily used it to differentiate between 
\WHILE programs and the functions they denote. In the case of problem
descriptions, we can the interpretation of a problem would be the solution 
that we would expect. 

\begin{example}
	$U = Cons(Nil, Cons(Nil, Cons(Nil, Nil)))$, then $\interpret[unary]{U} = 3$.
\end{example}
\begin{example}
	$L=Cons(4, Cons(8, Cons(2, Cons(-9, Nil))))$ is the problem description for
	sorting the list $4, 8, 2, -9$ if given to a sorting procedure, formally
	$\interpret[sort]{L}= Cons(-9, Cons(2, Cons(4, Cons(8, Nil))))$. When given
	to a procedure, that calculates the minimum, the interpretation would be that
	instead, so $\interpret[min]{L} = -9$.
\end{example}

In this light, a problem description becomes a small and domain specific 
programming language, with the algorithm that solves it being an interpreter.
\lineofthought{Partial evaluation for partial problem descriptions?}
\TODO
\subsection{Reductions}
\lineofthought{Reduction is efficient cross-compilation}
\TODO

\subsection{Hardness and completeness}
\lineofthought{Hardness is being universal for a complexity class}
\TODO
